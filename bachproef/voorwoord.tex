%%=============================================================================
%% Voorwoord
%%=============================================================================

\chapter*{\IfLanguageName{dutch}{Woord vooraf}{Preface}}%
\label{ch:voorwoord}

%% TODO:
%% Het voorwoord is het enige deel van de bachelorproef waar je vanuit je
%% eigen standpunt (``ik-vorm'') mag schrijven. Je kan hier bv. motiveren
%% waarom jij het onderwerp wil bespreken.
%% Vergeet ook niet te bedanken wie je geholpen/gesteund/... heeft

Het schrijven van deze bachelorproef markeert het einde van mijn opleiding Toegepaste Informatica aan de Hogeschool Gent. Deze periode was gevuld met leerkansen, uitdagingen en persoonlijke groei. Gedurende deze opleiding heb ik niet alleen mijn kennis in informatica verdiept, maar ook geleerd hoe ik deze kon toepassen in de realiteit.

\vspace{1em}

Deze bachelorproef kwam tot stand vanuit mijn interesse in data en AI en hoe dit op een slimme manier gebruikt kan worden. Ik onderzocht of klantsegmentatie kan helpen om verkoopvoorspellingen nauwkeuriger te maken met behulp van regressiemodellen.

\vspace{1em}

Graag wil ik mijn promotor Gilles Blondeel bedanken voor zijn begeleiding en waardevolle feedback gedurende de bachelorproef. Mijn co-promotor Elias De Smett ben ik bijzonder dankbaar voor zijn praktische ondersteuning en hulp bij de technische uitwerking van de proof-of-concept. Tot slot wil ik mijn stagementor, Mathieu Goossens, bedanken voor zijn begeleiding en de waardevolle inzichten die ik heb opgedaan tijdens het uitvoeren van mijn bachelorproef bij IPCOM NV.

\vspace{1em}

Daarnaast ben ik mijn vrienden en familie erg dankbaar voor hun voortdurende ondersteuning en aanmoediging. Hun aanwezigheid en geloof in mij hebben me keer op keer de moed gegeven om vooruit te blijven gaan.


