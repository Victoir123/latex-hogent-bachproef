%%=============================================================================
%% Inleiding
%%=============================================================================

\chapter{\IfLanguageName{dutch}{Inleiding}{Introduction}}%
\label{ch:inleiding}


\section{\IfLanguageName{dutch}{Probleemstelling}{Problem Statement}}%
\label{sec:probleemstelling}

Het voorspellen van sales en winst is een essentieel proces voor elk bedrijf dat strategische beslissingen wil nemen of financiële plannen wil opstellen. In de huidige tijd is het bijna onmogelijk om dit effectief te doen zonder gebruik te maken van data-analyse en andere technieken zoals klantsegmentatie. Bedrijven worden geconfronteerd met een steeds dynamischere markt, wat ervoor zorgt dat traditionele voorspellingsmethoden vaak tekortschieten.
 
\vspace{1em}

De doelgroep van dit onderzoek richt zich specifiek op managers en analisten in commerciële en financiële functies binnen bedrijven. Commercieel managers zijn verantwoordelijk voor het optimaliseren van verkoopstrategieën, terwijl financiële analisten gefocust zijn op het opstellen van nauwkeurige budgetten. Beide rollen zijn direct afhankelijk van betrouwbare en data gedreven voorspellingen om strategische beslissingen te ondersteunen.

\vspace{55mm}

\section{\IfLanguageName{dutch}{Onderzoeksvraag}{Research question}}%
\label{sec:onderzoeksvraag}

De centrale onderzoeksvraag van deze studie luidt: \textit{Kan klantsegmentatie op basis van aankoopgedrag de nauwkeurigheid van salesvoorspellingen verbeteren?}

Om deze vraag te beantwoorden, worden de volgende deelvragen onderzocht:

\begin{itemize}
    \item Wat is efficiënter bij het voorspellen van sales, forecastmethoden zoals SARIMA of regressiemodellen?
    \item Welke regressiemodellen presteren het best bij het voorspellen van sales?
     \item Welke clusteringtechnieken, zoals K-means, GMM, DBSCAN en K-Modes, leveren de meest betekenisvolle klantsegmenten op voor het verbeteren van salesvoorspellingen?
    \item Welke regressiemodellen presteren het best bij het voorspellen van sales in combinatie met een klantsegmentatie?
    \item Hoe kan klantsegmentatie effectief worden geïntegreerd in een regressiemodel voor salesvoorspellingen om de nauwkeurigheid te verbeteren?
\end{itemize}



\section{\IfLanguageName{dutch}{Onderzoeksdoelstelling}{Research objective}}%
\label{sec:onderzoeksdoelstelling}

Het eindresultaat van dit onderzoek is een proof-of-concept voor het bedrijf IPCOM NV, waarbij de nauwkeurigheid van salesvoorspellingen met en zonder klantsegmentatie wordt vergeleken. Het succes van deze bachelorproef wordt bepaald door het aantonen dat klantsegmentatie de precisie van de voorspellingen verbetert, gemeten aan de hand van maatstaven zoals Mean Squared Error (MSE), Mean Absolute Error (MAE) en Root Mean Squared Error (RMSE)."

\section{\IfLanguageName{dutch}{Opzet van deze bachelorproef}{Structure of this bachelor thesis}}%
\label{sec:opzet-bachelorproef}

De rest van deze bachelorproef is als volgt opgebouwd:

In Hoofdstuk~\ref{ch:stand-van-zaken} wordt een overzicht gegeven van de stand van zaken binnen het onderzoeksdomein, op basis van een literatuurstudie.

In Hoofdstuk~\ref{ch:methodologie} wordt de methodologie toegelicht en worden de gebruikte onderzoekstechnieken besproken om een antwoord te kunnen formuleren op de onderzoeksvragen.

In Hoofdstuk~\ref{ch:Proof of concept} wordt een proof of concept uitgewerkt en worden de deelvragen en de hoofdvraag aan de hand daarvan beantwoord.

In Hoofdstuk~\ref{ch:conclusie}, tenslotte, wordt de conclusie gegeven en een antwoord geformuleerd op de onderzoeksvragen. Daarbij wordt ook een aanzet gegeven voor toekomstig onderzoek binnen dit domein